
\documentclass[11pt]{report}

\usepackage{amsmath}
\usepackage{amssymb}
\usepackage{graphicx}
\usepackage{hyperref}
\usepackage[utf8]{inputenc}

\begin{document}

\title{Rapport sur les différents capteurs, mousses et cartes de contrôle}
\author{Auguste Sallen}
\date{}
\maketitle

\section{Capteurs de pression}

Les capteurs de pression doivent répondre à différents critères. D'après la 
documentation technique du capteur NumaTac, ils doivent remplir les conditions suivantes :

\begin{itemize}
  
  \item{Capteur de type ''pressure transducer''}
  
  \item{0-1 psi range (0-6,89 kPa)}
  
  \item{Référence à la pression atmosphérique}
  
  \item{0,2 \% de précision (au moins)}
  
  \item{Mesure de la pression statique et dynamique}
  
\end{itemize}

D'après des critères concernant notre cas d'étude, qui est une peau pour robot humanoïde :

\begin{itemize}
  
  \item{Range un peu plus élevée pour anticiper des chocs plus violents}
  
  \item{Capteur de petite taille}
  
  \item{Capteur de poids peu élevé}
  
\end{itemize}

Notre sélection de capteurs peut alors être orientée afin d'en retenir un 
minimum et de procéder à un choix.

Les capteurs qui ont été retenus sont les suivants (liens URL) :

\begin{itemize}
  
  \item{Pressure sensor 13 and 43 - 1 psi}
  
  \item{Pressure sensor 1220 - 1 psi}
  
  \item{Pressure sensor P-2000}
  
  \item{Pressure sensor MS5525DSO}
  
\end{itemize}

Il est alors possible de procéder à un choix de capteur. Celui-ci sera guidé par le 
cahier des charges énoncé précédemment. Tous les capteurs présents ont été 
grossièrement sélectionnés en fonction de ces critères, en particulier la range 
de pression et la taille du capteur. 

Tous les capteurs énoncés font en effet un volume d'1 cm$^3$ environ, ce qui est 
le plus petit pour cette range de pression.

Les deux premiers capteurs sont limités à 0-1 psi, soit 0-6,89 kPa. Ceci 
convient dans le cadre du capteur NumaTac, qui est ajouté sur un doigt 
robotique. Cependant, un doigt ne reçoit pas les mêmes impacts qu'un coude ou 
qu'un genou. Il est donc possible que même à trois, ils ne soient pas en mesure 
de prendre les mesures de pression sans saturer. Il est également possible que 
les surcharges de pression imposées aux capteurs soient destructives assez 
rapidement.

Les deux capteurs suivants sont des capteurs représentant une gamme dans 
laquelle il est possible de choisir le capteur que l'on souhaite. Il est donc 
possible de prendre dans ces gammes des capteurs correspondant aux besoins 
éprouvés.

Le capteur P-2000 proposant une gamme ayant plus d'échelons sur les range de pression 
mesurées, il sera retenu en priorité.

Un capteur de qualité acceptable serait alors de la dénomination 
P-2000-102G-15-BN. Sa range de pression est de 0-98,1 kPa.
Il peut donc mesurer des efforts allant jusqu'à 60 kg appliqués sur un 
coude (environ), s'il est présent en triple exemplaire sur le bras. 

Capteur retenu: P-2000-102G-15-BN.




\section{Carte de commande}

Il faut à présent faire un choix concernant la carte de commande pour le driver 
moteur. Cette carte doit pouvoir recevoir et sortir des tensions comprises entre 
-10 V et +10 V. 

Les cartes arduino sont les premières à venir à l'esprit: elles sont simples 
d'utilisation, faciles à trouver, leurs caractéristiques sont connues et elles 
sont directement compatibles avec des modes sur Matlab et Simulink.

Cependant, elles ne permettent pas de sortir la tension nécessaire sur les 
bornes I/O. Il est donc nécessaire de ce pencher sur un autre moyen que 
celui-ci. D'autres cartes sont directement compatibles avec Matlab et/ou 
Simulink, mais elles ne remplissent pas non plus les critères qui ont été posés.

Il est à noter qu'il serait possible d'utiliser ces cartes avec un montage 
amplificateur, ou sommateur. Cependant, cette méthode est plus complexe, demande 
d'avoir plus de matériel et augmente le volume et le poids requis par l'ensemble 
de commande. Elle est malgré tout envisageable dans la mesure où il ne s'agit 
encore que de tests

Il a été décidé d'utiliser l'outil xPC Target de Matlab. Cet outil permet 
de communiquer avec un ordinateur (et en particulier une carte de commande) via 
USB tout en lui transmettant des ordres en temps réel depuis l'ordinateur 
d'interface (celui sur lequel est utilisé Matlab). 

A l'aide de cet outil, il est possible de contrôler le driver en passant par une 
carte de type Analog I/O. 

Différents problèmes ont alors été rencontrés:

\begin{itemize}
  
  \item{L'outil xPC Target n'est pas compatible avec OSX ni Linux}
  
  \item{La solution serait d'utiliser Embedded Coder}
  
  \item{Les cartes trouvées utilisent un bus PCI: il faut donc un adaptateur}
  
  \item{Les cartes trouvées fonctionnent rarement sous OSX: il faut (sans doute) changer d'ordinateur}
  
\end{itemize}

Le plus simple, dans ces conditions, serait d'utiliser un PC ayant pour système 
d'exploitation Windows. Il serait alors possible d'utiliser xPC Target, ou 
encore les driver fournis directement avec les cartes Analog I/O. 

Embedded Coder traduit les codes écrits sous Matlab ou les blocs sous Simulink 
directement en C / C++. Son usage est donc limité aux cartes acceptant la 
programmation dans ces langages.

Plusieurs cartes semblant convenir ont par ailleurs été trouvées:

\begin{itemize}
  
  \item{HRP Interface Board 07-0003}
  
  \item{Analog I/O board 18AISS8AO8}
  
  \item{Analog I/O board XT1540}
  
  \item{Analog I/O board VCM-DAS-3}
  
  \item{Analog I/O board APC730}
  
  \end{itemize}

La première carte est celle qui a été retrouvée en tant que carte utilisée avec 
le driver moteur HRP qui est déjà possédé. Cette carte n'utilise cependant 
qu'une version ancienne de Windows et de Linux. Il s'agit malgré tout 
probablement de la carte qui est utilisée dans le robot humanoïde HRP, c'est 
donc ce genre de carte qu'il faut s'attendre à devoir utiliser par la suite dans 
notre cadre d'étude.

Les cartes suivantes répondent aux exigences de base, à ceci près qu'elle ne 
fonctionnent que sous Windows et Linux. Aucune carte ne fonctionnant avec un 
driver compatible OSX n'a été trouvée. 

La comparaison devra donc se faire sans critère. Au vu des caractéristiques, 
chacune des cartes semble convenir. De plus, rien ne les départage en 
particulier. Ce sera donc le prix qui sera déterminant.

\begin{itemize}
  
  \item{APC730 : 1560 euros}
%https://appleinsider.com/articles/17/01/13/install-an-inexpensive-usb-c-pci-e-card-in-a-mac-pro-for-full-usb-31-data-transfer-speeds

  \item{VCM-DAS-3A : 495 euros}
  
  \item{XT1540 : 525 euros}
  
  \item{18AISS8AO8 : no info}
%http://www.generalstandards.com/analogio4.php#12analogio
  
  \item{HRP IB 07-0003 : ~250 euros}
%http://www.osakac.ac.jp/labs/koeda/pdf/RSJ2008_kinugawa.pdf
%http://www.zuco.jp/pdt/ifb_hrp2.html
%http://www.zuco.jp/doc/ctg_hrp2_140401.pdf
%http://www.zuco.jp/doc/man_hrp2_140401.pdf
%http://www.zuco.jp/pdt/eol_hrp1.html
%http://www.zuco.jp/doc/man_hrp1_140401.pdf

\item{Arduino MEGA : 38,5 euros}

\item{Arduino UNO : 19 euros}

\item{Arduino Due : 34 euros}
  
  \end{itemize}



La décision la plus économique semble être celle de prendre une carte Arduino, 
une UNO suffisant amplement, et d'ajouter un montage amplificateur à sa 
sortie afin de doubler les 5 V qui sortent de la carte. Il est même possible 
qu'il nous soit inutile de contrôler le moteur sur toute sa plage d'utilisation, 
et qu'une commande sur la moitié de la vitesse max soit suffisante.

Dans le cadre d'un usage dans un robot humanoïde, le choix de prendre le modèle 
HRP semble être le meilleur, même s'il nécessite un apprentissage 
supplémentaire.


\section{Mousse utilisée}




\section{Bibliographie et liens}

\subsection{Capteurs de pression}


  \url{http://www.te.com/commerce/DocumentDelivery/DDEController?Action=srchrtrv&DocNm=13_and_43_1psi&DocType=Data+Sheet&DocLang=English&DocFormat=pdf&PartCntxt=CAT-BLPS0052}
  
  \url{http://www.te.com/commerce/DocumentDelivery/DDEController?Action=srchrtrv&DocNm=1220_1PSI&DocType=Data+Sheet&DocLang=English&DocFormat=pdf&PartCntxt=CAT-BLPS0046}
  
  \url{https://www.nidec-copal-electronics.com/e/catalog/pressure-sensor/p-2000.pdf}
  
  \url{http://www.te.com/commerce/DocumentDelivery/DDEController?Action=srchrtrv&DocNm=MS5525DSO&DocType=DS&DocLang=English}
  


\end{document}
			