
\documentclass[11pt]{report}

\usepackage{amsmath}
\usepackage{amssymb}
\usepackage{graphicx}
\usepackage{hyperref}
\usepackage[utf8]{inputenc}

\begin{document}

\title{Rapport sur les différents capteurs, mousses et cartes de contrôle}
\author{Auguste Sallen}
\date{}
\maketitle

\section{Capteurs de pression}

Les capteurs de pression doivent répondre à différents critères. D'après la 
documentation technique du capteur NumaTac, ils doivent remplir les conditions suivantes :

\begin{itemize}
  
  \item{Capteur de type ''pressure transducer''}
  
  \item{0-1 psi range (0-6,89 kPa)}
  
  \item{Référence à la pression atmosphérique}
  
  \item{0,2 \% de précision (au moins)}
  
  \item{Mesure de la pression statique et dynamique}
  
\end{itemize}

D'après des critères concernant notre cas d'étude, qui est une peau pour robot humanoïde :

\begin{itemize}
  
  \item{Range un peu plus élevée pour anticiper des chocs plus violents}
  
  \item{Capteur de petite taille}
  
  \item{Capteur de poids peu élevé}
  
\end{itemize}

Nous pouvons alors orienter notre sélection de capteurs afin d'en retenir un 
minimum et de procéder à un choix.

Les capteurs qui ont été retenus sont les suivants (liens URL) :

\begin{itemize}
  
  \item{Pressure sensor 13 and 43 - 1 psi}
  
  \item{Pressure sensor 1220 - 1 psi}
  
  \item{Pressure sensor P-2000}
  
  \item{Pressure sensor MS5525DSO}
  
\end{itemize}

Nous pouvons alors procéder à un choix de capteur. Celui-ci sera guidé par le 
cahier des charges énoncé précédemment. Tous les capteurs présents ont été 
grossièrement sélectionnés en fonction de ces critères, en particulier la range 
de pression et la taille du capteur. 

Tous les capteurs énoncés font en effet un volume d'1 cm$^3$ environ, ce qui est 
le plus petit pour cette range de pression.

Les deux premiers capteurs sont limités à 0-1 psi, soit 0-6,89 kPa. Ceci 
convient dans le cadre du capteur NumaTac, qui est ajouté sur un doigt 
robotique. Cependant, un doigt ne reçoit pas les mêmes impacts qu'un coude ou 
qu'un genou. Il est donc possible que même à trois, ils ne soient pas en mesure 
de prendre les mesures de pression sans saturer. Il est également possible que 
les surcharges de pression imposées aux capteurs soient destructives assez 
rapidement.

Les deux capteurs suivants sont des capteurs représentant une gamme dans 
laquelle il est possible de choisir le capteur que l'on souhaite. Il est donc 
possible de prendre dans ces gammes des capteurs correspondant aux besoins 
éprouvés.

Le capteur P-2000 proposant une gamme ayant plus d'échelons sur les range de pression 
mesurées, il sera retenu en priorité.

Un capteur de qualité acceptable serait alors de la dénomination 
P-2000-102G-15-BN. Sa range de pression est de 0-98,1 kPa.
Il peut donc mesurer des efforts allant jusqu'à 60 kg appliqués sur un 
coude (environ), s'il est présent en triple exemplaire sur le bras. 

Capteur retenu: P-2000-102G-15-BN.




\section{Carte de commande}

Nous allons faire un choix concernant la carte de commande pour le driver 
moteur. Cette carte doit pouvoir recevoir et sortir des tensions comprises entre 
-10 V et +10 V. 


\section{Bibliographie et liens}

\subsection{Capteurs de pression}


  \url{http://www.te.com/commerce/DocumentDelivery/DDEController?Action=srchrtrv&DocNm=13_and_43_1psi&DocType=Data+Sheet&DocLang=English&DocFormat=pdf&PartCntxt=CAT-BLPS0052}
  
  \url{http://www.te.com/commerce/DocumentDelivery/DDEController?Action=srchrtrv&DocNm=1220_1PSI&DocType=Data+Sheet&DocLang=English&DocFormat=pdf&PartCntxt=CAT-BLPS0046}
  
  \url{https://www.nidec-copal-electronics.com/e/catalog/pressure-sensor/p-2000.pdf}
  
  \url{http://www.te.com/commerce/DocumentDelivery/DDEController?Action=srchrtrv&DocNm=MS5525DSO&DocType=DS&DocLang=English}
  


\end{document}
			